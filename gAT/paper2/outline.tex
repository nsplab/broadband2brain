\documentclass[]{article}

\usepackage{geometry}
\usepackage{enumerate}

\begin{document}
\begin{abstract}
\begin{itemize}
\item Abstract
\begin{itemize}
\item Goal: sample from human brain with high density of electrodes
\item Need to find number of spikes and spike times
\item Current methods need to sample at about 20 KHZ (?): power and data bandwidth are too high
\item present method that only requires sampling at 10 - 30 Hz (?)
\end{itemize}
\end{itemize}
\end{abstract}

\section{Introduction}
\begin{itemize}
\item Motivation (Figure 1a (plot of battery life))
\begin{itemize}
\item Nyquist sampling of neural signals requires 5 - 30 KHz
\item Want to be able to place electrodes on brain with high density
\item Resulting energy / transmission needs are too high
\item Need to be able to sample slower
\end{itemize}
\item Reconstructing signals with finite rate of innovation
\item Analog Thresholding
\item 3 Failure Modes of AT
\begin{itemize}
\item Spike time resolution
\item Missed multiple spike
\item Loss of spike shape
\end{itemize}
\item gAT addresses first two failure modes
\end{itemize}

\section{Methods}

\begin{itemize}
\item gAT (might want to not explain 1, 2-spike independently and just list as special cases?) (figure 2)
\begin{itemize}
\item one-spike
\item two-spike
\item general
\begin{itemize}
\item TODO: how are the analog thresholds computed?
\item compute successive integrals (explain circuitry?)
\item Predict with as one spike initially
\item Increase the number of spikes if unable to predict next succesive integral correctly
\item reconstruction can be done remotely
\end{itemize}
\end{itemize}
\end{itemize}

\section{Results}

\begin{itemize}
\item able to determine spike times (figure 2, sort of)
\item able to determine number of spikes (figure 3)
\item Monkey simulations
\begin{itemize}
\item Tuning curve (figure 5a)
\item phase and distribution error (figure 5b)
\item LFP
\item error in preferred direction (figure 6a)
\end{itemize}
\item ``Saturation effect'' (figure 6bc)
\end{itemize}

\section*{Figures}

\begin{enumerate}
\item
\begin{enumerate}[(a)]
\item Battery life over time given density of electrodes (where was the original data?)
% Bandwidth?
% Chandrakesan MIT (microwatts per second for sampling rate for existing DSP)
% Reed Harrison (AT hardware) (IEEE Paper)
% Applications of analog threshold
% Where introduced
% AT recently adopted (common use in offline analysis for spike detection)
% Plexon - company that makes hardware + offline software for neural
% AT widely used offline + online spike detection
% More recently RH implemented AT for efficient sampling of spikes
% spike times are registered digitally when amplitude of spike passes threshold
% RH check output of comparator at 1000 Hz
% AT drops from Nyquist sampling to sampling output of comparator at 1 KHz

% Conventional AT:
% Benefit:
%  - reduce sample rate
%  - simple to implement
%  - sampling rate approaches rate of innovation
% 

% conventionally thought nyquist rate was fundamental limit
% compressive sensing wants sub-nyquist sampling methods
% nyquist required for arbitrary band-limited signal
% neural signals are parameterized
% spikes are similar; vary in amplitude and time

% Kusuma Ultra-wideband
% Reconstruct shape and time of pulse
% Shape of pulse for a given neuron is relatively constant
% http://news.brown.edu/pressreleases/2013/02/wireless power consumption
% MIT Compressive sensing - power consumption
% IEEE Spectrum compressive sensing review
% Model-based compressive sensing (Rice)

% Juwan Yoo (Emami)
% Strong fix conditions
% RIP properties (compressive sensing)
% on-board: sample and reconstruction
% telemetry: send signals with transmitter (transmission of signal to off-board computer)
% apply to both on/off-board
% don't just say trying to reduce bandwith
%    ++channels -> more data for transmission and downstream processing
% power consumption
%   sample
%   processing
%   transmission
% pacemaker: example where processing is on board
% AICD: electrocardiogram sampled + processed on board

% Spike acquisition systems take both approaches
% on-board: hermes B (on board) (Stanford hardware) Hermes C (wireless component)
% FETZ: UW Seattle
% both classes of methods benefit from decrease in sampling rate / efficient sampling
% on-board with decreased sampling rate reduces data that needs to be processed
% telemetry needs to send less data

% Very low-sampling rate allows hardware to be shut down
% Can reduce duty-cycle of on-board / transmission
% Alternatively can send large data with same power (same amount of neural information)

% Real time reconstruction of signals
% Brain-Machine Interface
% Seizure detection
% closed-loop brain-machine interface
%  seizure management
%  stroke rehabilitation
%  treatment of movement disorders
%  prosthetic
%  mood disorder (PTSD)
%  motor disorder (parkinsons)
%  epilepsy

% Growing interest in solving problems with brain function that require real-time reconstruction of neural spikes.
% People are trying to do whole brain recording for these problems.
% Dense electrode arrays
% retinal ganglia cells recorded (Neuron neurotechnique) salamander retina 
% real-time basic science and clinical applications with whole brain recording require methods to be developed
% 8 microwatts to survive for 10 years on 1 battery (chandrakesan MIT)
% Hardware complexity
% Real-time operation of algorithm
% On-board and transmission bandwidth

% Near end of intro: paper describes new sampling methods and application to the recording of spiking neural data
% methods generalizes AT to allow for super-resolution sampling and multiplicity of spikes
% sampling method based on concepts in finite rate of innovation theory
% show that certain features of neural activity can be reliably estimated at sampling rates of 20 Hz
% paper explores regimes of neural signal processing where super-res and multiplicity are important
% Keeps simplicity and real-time; allows super-res and multiplicity

% Neuron Neurotechnique
% Nature Neuroscience: techniques
% Collect key articles for review of FRI and compressive sensing

% andrew richardson (journal of neurophysiology)

\item electrodes in brain + sample of spiking
\item zoom in of a spike
\end{enumerate}
\item diagram of AT, gAT-1, gAT-2
\item
\begin{enumerate}[(a)]
\item Fraction of Intervals with n Spikes
\item Aggregate Fraction of Valid Intervals
\item Pairwise difference of fraction of valid intervals
\item ROC 15 ms
\item ROC 50 ms
\item ROC 100 ms
\end{enumerate}
\item
\begin{enumerate}
\item Tuning 15 ms
\item Tuning 50 ms
\item Tuning 100 ms
\end{enumerate}
\item
\begin{enumerate}
\item Phase error
\item Distribution error
\end{enumerate}
\item
\begin{enumerate}
\item error in preferred direction (Move to figure 5c?)
\item LFP (Saturation Effect 1)
\item LFP (Saturation Effect 2)
\end{enumerate}
\end{enumerate}

\section*{Miscellaneous}
\begin{itemize}
\item What is a real spike?
\end{itemize}

\end{document}
